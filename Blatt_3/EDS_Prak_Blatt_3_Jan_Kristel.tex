\documentclass{article}
\usepackage{fontenc}
\usepackage[ngerman]{babel}
\usepackage[utf8]{inputenc}
\usepackage{graphicx}
\usepackage{grffile}
\usepackage{subcaption}
\usepackage[export]{adjustbox}
\graphicspath{ {../pics/Blatt 2} }
\usepackage{alphalph}
\DeclareUnicodeCharacter{2028}{\linebreak}
\usepackage{hyperref}
\usepackage{listings}
\usepackage{color}

\definecolor{dkgreen}{rgb}{0,0.6,0}
\definecolor{gray}{rgb}{0.5,0.5,0.5}
\definecolor{mauve}{rgb}{0.58,0,0.82}

\lstset{frame=tb,
  language=Java,
  aboveskip=3mm,
  belowskip=3mm,
  showstringspaces=false,
  columns=flexible,
  basicstyle={\small\ttfamily},
  numbers=left,
  numberstyle=\tiny\color{gray},
  keywordstyle=\color{blue},
  commentstyle=\color{dkgreen},
  stringstyle=\color{mauve},
  breaklines=true,
  breakatwhitespace=true,
  tabsize=3
}
\usepackage{datetime}
\newdateformat{myformat}{\THEDAY{ten }\monthname[\THEMONTH], \THEYEAR}

\begin{document}
		\begin{titlepage}
		\centering
		{\scshape\LARGE
			Ereignisdiskrete Systeme
			\par}
		\vspace{1.5cm}
		{\huge\bfseries Praktikum Blatt 3 - Petri-Netze\par}
		\vspace{1.5cm}
		{\LARGE\itshape Jan Kristel, Alexandra Moritz\par}
		\vfill
			Aufsicht von Frau Rembold\par
			
		\vfill	
			{\large \today \par}	
		
	\end{titlepage}
	
	\tableofcontents
	\newpage
	
	\section{Petri-Netz}
	\section{Anfangsmarkierung}
	\section{Inzidenzmatrix}
	\section{Erreichbarkeitsgraph}
	\section{Netzeigenschaften}
		\subsection*{Erreichbarkeit}
		\subsection*{Deadlock}
		\subsection*{Lebendigkeit}
		\subsection*{Umkehrbarkeit}
		\subsection*{Konfliktfreiheit}
		\subsection*{Beschränktheit}
	\section{Schaltvektor}
	\section{Nachweis für den Schaltevektor und die Schaltsequenz $\sigma$}
	\section{Modellierung mit Petri06}
	\section{Modellierung mit PetriEdiSim}
	\section{Vor- und Nachteile von Petri06 und PetriEdiSim}
	
\end{document}